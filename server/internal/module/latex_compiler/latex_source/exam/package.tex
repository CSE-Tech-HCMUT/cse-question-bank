\documentclass[11pt]{./internal/module/latex_compiler/latex_source/exam/examdesign}
\usepackage{amsmath,amsxtra,latexsym, amssymb, amscd}
\usepackage[utf8]{vietnam}
\usepackage{pstricks-add}
\usepackage{graphicx}
\usepackage{wrapfig}
\usepackage{enumitem}
\usepackage{times} 
\usepackage{ifthen} 
\usepackage{arydshln} % for dashline table
\setlength\dashlinedash{0.5pt}
\setlength\dashlinegap{1.5pt}
\setlength\arrayrulewidth{0.3pt}
\usepackage{mathtools}
\DeclarePairedDelimiter\ceil{\lceil}{\rceil}
\DeclarePairedDelimiter\floor{\lfloor}{\rfloor}
\usepackage{multicol}
\usepackage{multirow}
\usepackage{tikz}% http://ctan.org/pkg/pgf
\usepackage{listings}
\usepackage{xcolor}
\usepackage{float}

% Tikz
\usetikzlibrary{automata,arrows,positioning, chains, shapes.callouts, calc, trees}

\tikzstyle{mnode}=[circle, draw, fill=black, inner sep=0pt, minimum width=4pt]
\tikzstyle{thinking} = [draw=blue, very thick]
\tikzstyle{subtree} = [draw=blue, fill=blue!10, shape border uses incircle,
  isosceles triangle,shape border rotate=90, inner sep=.8pt, minimum size=3ex]
\tikzstyle{cgreen} = [color={rgb:red,0.0;green,0.42;blue,0.24}]

\usetikzlibrary{shapes.geometric,arrows,fit,matrix,positioning}
\usetikzlibrary{matrix,backgrounds}

\tikzstyle{process} = [rectangle, thick, draw, minimum size = 1mm, font = \tiny]
\tikzstyle{decision} = [diamond, thick, draw, minimum size = 1mm, font = \tiny, aspect = 2]
\tikzstyle{io} = [trapezium, trapezium left angle=70, trapezium right angle=110,  thick, draw, minimum size = 1mm, font = \tiny]
\tikzstyle{startstop} = [rectangle, rounded corners=3pt, thick, draw, minimum size = 1mm, font = \tiny]
\tikzstyle{arrow} = [->, thick]
\tikzstyle{arrow1} = [dashed,->, thick]
\tikzstyle{connector} = [circle, draw, thick, minimum size = 2mm]
\tikzset{font = \tiny}

% Lstlistings
\lstset{
language=Python, 
basicstyle=\scriptsize\ttfamily\color{black},
}


\lstdefinestyle{ANTLR}{
    basicstyle=\scriptsize\ttfamily\color{black},%
    breaklines=true,%                                      allow line breaks
    moredelim=[s][\color{black}\ttfamily]{'}{'},% single quotes in green
    moredelim=*[s][\color{black}\ttfamily]{options}{\}},%  options in black (until trailing })
    commentstyle={\color{black}\itshape},%                  gray italics for comments
    morecomment=[l]{//},%                                  define // comment
    emph={%
        STRING%                                            literal strings listed here
        },emphstyle={\color{black}\ttfamily},%              and formatted in blue
    alsoletter={:,|,;},%
    morekeywords={:,|,;},%                                 define the special characters
    keywordstyle={\color{black}},%                         and format them in black
}
\usepackage[linesnumbered]{algorithm2e}

\usepackage{./internal/module/latex_compiler/latex_source/exam/dethi} %Gói lệnh cho đề thi Việt Nam

\usepackage{geometry} % Required for adjusting page dimensions and margins

\IncludeFromFile{chunks/chunks}

\geometry{
% 	paper=a4paper, % Paper size, change to letterpaper for US letter size
% 	top=0cm, % Top margin
	bottom=1cm, % Bottom margin
% 	left=.8cm, % Left margin
% 	right=1cm, % Right margin
% 	headheight=0.75cm, % Header height
% 	footskip=1.5cm, % Space from the bottom margin to the baseline of the footer
% 	headsep=0.75cm, % Space from the top margin to the baseline of the header
% % 	showframe, % Uncomment to show how the type block is set on the page
}

\Fullpages %Định dạng trang đề thi
\ContinuousNumbering %Đánh số liên tục các bài thi
\ShortKey 
%\OneKey %Lệnh chỉ in ra 1 bản đáp án
%\NoKey %Lệnh không in ra phần đáp án
\NumberOfVersions{1} %20 là số bài thi khác nhau được in ra
\SectionPrefix{\relax }%\bf Phần \Roman{sectionindex}. \space}
\setrandomseed{7}
\tentruong{TRƯỜNG ĐH BÁCH KHOA - ĐHQG-HCM}
\tenkhoa{KHOA KH \& KT MÁY TÍNH}
\tenkythi{THI CUỐI KỲ}%{ĐỀ THI LẠI}%%{ĐỀ CHÍNH THỨC}
\tenmonhoc{Nguyên lý ngôn ngữ lập trình}
\mamonhoc{CO3005}
\thoigian{90 phút}
\madethi{2320}
\hocky{2}
\namhoc{2023-2024}
\ngaythi{17-05-2024}
\ghichu{\textbf{Ghi chú:}
\\- KHÔNG được phép dùng tài liệu.
\\- Sinh viên làm bài trên phiếu trả lời trắc nghiệm. 
\\- Các câu hỏi chỉ có 1 đáp án đúng hoặc không có đáp án đúng. Nếu không có đáp án đúng, sinh viên chọn đáp án E.
\\- Sinh viên nộp đề cùng với phiếu trả lời trắc nghiệm sau khi thi.
}
\tieudetracnghiem
%\tieudethiviet
\tieudedapan
%\tieudetren
\tieudeduoi
% \daungoac{[}{]}%Dấu quanh phương án trả lời: {(}{)};{}{.};{}{)}
%\chuphuongan{\alph}%Ký tự cho các phương án
%\chuphuongan{\arabic}%\Roman%\roman%kể cả số cho các phương án
\chucauhoi{Câu} %Chữ trước các số câu hỏi
\mauchu{black}
\socauhoi{60}
\sotrang{10}
\setlength{\baselineskip}{0truept}
\def\v#1{\overrightarrow{#1}}
%\NoRearrange
\newcommand{\nn}{\mathbb{N}}
\newcommand{\zz}{\mathbb{Z}}
\newcommand{\rr}{\mathbb{R}}
\newcommand{\cc}{\mathbb{C}}